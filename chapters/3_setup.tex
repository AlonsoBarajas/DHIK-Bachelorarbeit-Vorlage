\chapter{Setup}
\label{chap:3_setup}

\lipsum[2-4]

\section{How to insert an image}

To insert an image in \LaTeX{}, it is important to call the $\backslash begin\{figure\}$  environment.\autoref{fig:my_label} shows the result to inserting an image.

\begin{figure}[h]
    \centering
    \includegraphics[scale = 0.5]{example-image}
    \caption{Inserting an image in \LaTeX}
    \label{fig:my_label}
\end{figure}

\subsection{Inserting subimages}

This template has the \textit{subcaption} package which lets you create subfigures as shown below. \autoref{fig:subfigA}, \autoref{fig:subfigB} and \autoref{fig:subfigC} are contained inside \autoref{fig:subfigure}


\begin{figure}[h]
    \begin{subfigure}{0.3\textwidth}
        \centering
        \includegraphics[width = \textwidth]{example-image-a}
        \caption{This is subfigure A}
        \label{fig:subfigA}
    \end{subfigure}
    ~ 
    \begin{subfigure}{0.3\textwidth}
        \centering
        \includegraphics[width = \textwidth]{example-image-b}
        \caption{This is subfigure B}
        \label{fig:subfigB}
    \end{subfigure}
     ~ 
    \begin{subfigure}{0.3\textwidth}
        \centering
        \includegraphics[width = \textwidth]{example-image-c}
        \caption{This is subfigure C}
        \label{fig:subfigC}
    \end{subfigure}
    
    \caption{This is the whole figure}
    \label{fig:subfigure}
\end{figure}